% Brian Mc George - MCGBRI004
% Jacques Heunis - HNSJAC003
% Timothy Gwynn - GWNTIM001
% Due: 31-07-2015
% Stage One - Software Engineering  - CSC3003S
\documentclass[a4paper,10pt]{article}
%**************************************************************************************************
% PACKAGES
%**************************************************************************************************
\usepackage{amsmath, amsthm, amsfonts, amssymb}
\usepackage{graphicx,color}
\usepackage{bm}	
\usepackage{float}
\usepackage{caption, subcaption}
%\usepackage{vector}

%**************************************************************************************************
% DEFAULT SETTINGS
%**************************************************************************************************
\marginparwidth -20 true pt    % Width of marginal notes.
\oddsidemargin  -10 true pt       % Note that \oddsidemargin=\evensidemargin
\evensidemargin -10 true pt
\topmargin -0.5 true in        % Nominal distance from top of page to top of
\textheight 9.75 true in         % Height of text (including footnotes and figures)
\textwidth 7 true in        % Width of text line.
\parindent=10pt                  % Do not indent paragraphs
\parskip= 1 ex
\columnseprule = 0.1pt
\footskip = 30 true pt
\hoffset = -0.1 true in
\voffset = -0.1 true in
\abovedisplayskip 1 true pt
\abovedisplayshortskip 1 true pt
\topsep 0 true pt
\newcommand*\varhrulefill[1][0.4pt]{\leavevmode\leaders\hrule height#1\hfill\kern0pt}

%**************************************************************************************************
% DOCUMENT DETAILS
%**************************************************************************************************


%**************************************************************************************************
% MAIN DOCUMENT 
%**************************************************************************************************

\begin{document}
\begin{titlepage} \begin{center} 
		\textsc{\LARGE University of Cape Town}
		\\[1.5cm] \textsc{\Large Software Engineering Stage One\\CSC3003S}
		\\[0.5cm]
		\noindent\rule[0.4mm]{\textwidth}{0.1mm}
		\\[0.4cm] { \huge \bfseries Tempest Trace \\[0.4cm] }
		\noindent\rule[0.4mm]{\textwidth}{0.1mm}
		\\[1cm]
		\begin{minipage}[t]{0.4\textwidth}
		\begin{flushleft}\large \emph{Authors:}\\ Brian Mc George - MCGBRI004 \\ Jacques Heunis - HNSJAC003 \\ Timothy Gwynn - GWNTIM001\end{flushleft}
		 \end{minipage} \begin{minipage}[t]{0.4\textwidth} 
		\begin{flushright} \large \emph{Supervisor:} \\ Assoc. Prof.~Patrick Marais\\patrick@cs.uct.ac.za\end{flushright}
		\begin{flushright} \large \emph{Tutor:} \\ Codie Roelf\\Codie.Roelf@alumni.uct.ac.za\end{flushright}
		 \end{minipage} \vfill {\large \today}
		\end{center}
		\end{titlepage}
\newpage
\tableofcontents
\newpage

\section{Risk Assessment}
\subsection{Art assets not delivered}
\textbf{Category:} Art assets\\
\textbf{Probability:} Medium\\
\textbf{Impact:} Critical
\\\textbf{Consequences}\\
This would result in lower graphical quality in the game.
Animations found on the internet would have to be modified to work with the game.
\smallskip\\\textbf{Mitigation}\\
Keep in contact with the artists to get regular progress reports and to ensure that they are working on the deliverables. Investigate alternative assets to be used if the requested assets are not delivered.
\smallskip\\\textbf{Monitoring}\\
Track asset delivery timeline and check if there are assets which are past their expected delivery date.
\smallskip\\\textbf{Management}\\
Reduce the scope of the project to allow time to find art assets online and to re-purpose assets and animations to work with the game.

\subsection{Data loss}
\textbf{Category:} Development\newline
\textbf{Probability:} Low\newline
\textbf{Impact:} Critical
\\\textbf{Consequences}\\
If data is lost it will result in that work having to be re-done. This will delay project progress and put it behind schedule. Certain features may have to be cut in order to get back on schedule to complete the project on time.
\smallskip\\\textbf{Mitigation}\\
The project will be pushed onto a remote repository at the University of Cape Town (UCT) and each member will keep an up-to-date copy of the project on their respective machines. Each team member must push changes to the remote repository on a regular basis and pull the most recent version from the repository if other members have made a change.
\smallskip\\\textbf{Monitoring}\\
Keep track of the number of commits that the remote repository at UCT is behind that of the local machine and the number of commits that the local machine is behind the remote repository. The greater the number the higher the probability of data loss.  
\smallskip\\\textbf{Management}\\
Pull the most recent version of the project from the remote repository or a different members machine and continue working from that point. If all the data is lost from all the locations, then the project will have to be restarted with a much smaller scope.

\subsection{Underestimated complexity}
\textbf{Category:} Development\newline
\textbf{Probability:} Medium\newline
\textbf{Impact:} Marginal
\\\textbf{Consequences}\\
The project will likely fall behind schedule as it will take extra time to develop the complex logic structures. 
\smallskip\\\textbf{Mitigation}\\
Each member should update their task progress regularly and ask for assistance from the other group members if battling to complete a given task.
\smallskip\\\textbf{Monitoring}\\
Keep track of task progress for each team member.
\smallskip\\\textbf{Management}\\
The scope of the project may have to be reduced to allow the project to be completed on time. Pair programming techniques may be used to make it easier to complete the complex part of the system as ideas can be bounced off one another and the pair can debug a problem together.

\subsection{Loss of a team member}
\textbf{Category:} Development\newline
\textbf{Probability:} Low\newline
\textbf{Impact:} Critical
\\\textbf{Consequences}\\
The tasks assigned to the member who has fallen out will have to be absorbed by the other group members. The work load on the other members will increase. Project progression will be slowed as a result.
\smallskip\\\textbf{Mitigation}\\
Each member should voice their concerns and indicate early on if they are going to be away for a given time in the development process.
\smallskip\\\textbf{Monitoring}\\
Regular communication between team members and track task progress of each member to ensure that each member is making progress on their allocated task.
\smallskip\\\textbf{Management}\\
The scope of the project will have to be reduced to allow the project to be completed on time. 

\subsection{Sound assets not delivered}
\textbf{Category:} Art assets\newline
\textbf{Probability:} Medium\newline
\textbf{Impact:} Negligible
\\\textbf{Consequences}\\
This would result in lower sound quality in the game. The game would not be as original and distinct.
\smallskip\\\textbf{Mitigation}\\
Keep in contact with the sound artists to get regular progress reports and to ensure that they are working on the deliverables.
\smallskip\\\textbf{Monitoring}\\
Track asset delivery timeline and check if there are assets which are past their expected delivery date.
\smallskip\\\textbf{Management}\\
The sound effects required are relatively common and most could be sourced quite easily online.

\subsection{Game is not enjoyable to play}
\textbf{Category:} Gameplay\newline
\textbf{Probability:} Low\newline
\textbf{Impact:} Critical
\\\textbf{Consequences}\\
This would result in a low return on investment for the project.
\smallskip\\\textbf{Mitigation}\\
Regular play-testing of the game and provide feedback to the team on changes to make in the next iteration.
\smallskip\\\textbf{Monitoring}\\
Keep track of play-test feedback from team members and external play testers.
\smallskip\\\textbf{Management}\\
Rework or remove elements of the game-play that hinder enjoyment and possibly increase the scope of the project to add additional diversity to the game. 

\subsection{Game has performance issues}
\textbf{Category:} Development\newline
\textbf{Probability:} Medium\newline
\textbf{Impact:} Marginal
\\\textbf{Consequences}\\
This would cause the game to be less fun to play and make it harder for the player to achieve flow.
\smallskip\\\textbf{Mitigation}\\
Review and re-factor code on regular basis to ensure code ad-hears to software design patterns and is well optimised.
\smallskip\textbf{Monitoring}\\
Keep track of minimum, maximum and average frame rate during play testing.
\smallskip\\\textbf{Management}\\
Do an in-depth analysis to determine where the inefficiencies reside and re-design that aspect or cut it from the game if it cannot be optimised.

\section{Roles}
\subsection{Fixed Roles}
\subsubsection{Jacques Heunis}
\textbf{Team leader}\\
The team leader is in charge of coordinating all the team members and ensuring that the project goal remains in focus. He is also in charge of identifying problems in the project and initiating steps to deal with them.
\smallskip\\
\textbf{Lead developer}\\
The most experienced developer, often consulted if other team members require assistance with their code, ensures that the team's code is consistent and compatible within itself.
\smallskip\\
\textbf{Point of contact}\\
Point of contact is in charge of managing communication with the external project members such as the visual and sound artists.

\subsubsection{Timothy Gwynn}
\textbf{Lead game designer}\\
Responsible for game design and input regarding game-play and mechanics. Integrates art and sound into game.
\smallskip\\
\textbf{Lead level designer}\\
In charge of overall level design; ensures that the level supports the game-play options provided by the team. Is also responsible for the robustness of the level and fixing bugs in the game world.

\subsubsection{Brian Mc George}
\textbf{Lead systems analyst}\\
The lead systems analyst is in charge of maintaining all software documentation for the project and managing the development of the systems architecture.
\smallskip\\
\textbf{Lead artificial intelligence (AI) developer}\\
Responsible for creating the AI which controls the non-player characters within the game. Must work closely with the level designer to ensure that the AI is equipped to handle any proposed game world changes

\subsection{Rotating roles}
\textbf{Quality assurance}\\
All team members will fulfil this role at some point in ensuring that both their work and the rest of the team's work is of a high quality, is bug free and is consistent with the rest of the project.
\smallskip\\
\textbf{Systems integrator}\\
The systems integrator is in charge of integrating the various systems produced by individual team members into the overall project. This role will often change depending on which team members have the most knowledge on how each system fits into the overall project.

\section{Scope}
\subsection{Overview, purpose and stakeholders}
The project is a first-person running game whereby the player competes against the clock and against another player to get from the start to the end of a parkour / free running track while trying to keep their movements smooth and free flowing. The player is tasked with traversing an obstacle course in as short a time as possible while avoiding enemies who would try to slow down or eliminate the player. In addition to this, the game will feature 2-player local multiplayer in which the players are pitted against each other to see who can complete the course the fastest. The multiplayer mode will also allow players to interact with each other, letting players try to slow their opponent down to get an edge.
\smallskip\\
The purpose of our project is to complete a game that, when played, will provide the user with an enjoyable experience. The design of the game is orientated around making the player feel empowered, creating enjoyment and fun by a sort of power-trip sensation rather than through challenge or in-depth story telling. This sensation is triggered by the way in which the player's avatar is made to act out difficult but flowing motions without requiring particularly complex input, thus making the player feel as they might if they themselves were able to move in the way depicted by the game.
\smallskip\\
Stakeholders involved in the project are the development team, the project supervisor and the players. \\
The development team consists of the software development team, the art team, and a sound designer. \\
The project supervisor is Assoc. Prof. Patrick Marais, a lecturer at UCT. \\
The players are all the people who will eventually play the game, including (but not limited to) the development team, friends or family of the development team, the project supervisors and other development teams.

\subsection{SMART Goals} \label{goals}
The functionality goals can be roughly split up into 3 categories: Player motion, Enemy interaction, and Game world:\\
\textbf{Player motion}
\begin{itemize}
	\item The player can run forwards, backwards and strafe to either side
	\item The player can jump over small obstacles
	\item The player can climb onto small-to-medium sized obstacles
	\item While running, the player can slide along the ground to get under low-hanging obstacles
	\item The player's avatar sports animations appropriate to their movement and interaction with the world
	\item The player will respawn if they get stuck or fail. The respawn location will update as the player progresses through the game.
\end{itemize}
\textbf{Enemy interaction}
\begin{itemize}
	\item There are two types of enemies, ones that simply chase down the player, and ranged enemies who try to get a clear line-of-sight and eliminate the player
	\item While the player is in sight, enemies will chase in an attempt to catch and subdue the player
	\item The player can place objects throughout the world which will distract enemies or obstruct their vision and prevent them from continuing the chase
\end{itemize}
\textbf{Game world}
\begin{itemize}
	\item The game world is a clean, brightly coloured 3D environment, viewed from a first person perspective
	\item The game supports 2-player local multiplayer where players play against each other competitively
	\item The game features context-sensitive audio feedback which helps inform the player of their situation via both sound effects and background music
	\item There will only be one level but it will take a minimum of 10 minutes to complete such that the player can properly immerse themselves in the game
    \item A level includes an introductory narrative to give the player context.
    \item A level flyover occurs during the introductory narrative to allow the player to plan their desired route.
\end{itemize}

\subsection{Inputs, outputs, and performance}
The inputs, in the form of buttons and thumbsticks, come from each player's game controller and will cause their character to react in a pre-defined manner. The input from the controllers will also be used to navigate the game's main menu, pause menu and end-level screen. The outputs come in the form of player motion as described in section \ref{goals} and various user interface screens. For example, clicking a button in the main menu may start the game or may lead to a new menu. Once a level is completed, the end-level user interface will show with information regarding the play-through.\smallskip\\The game will have to run with at least 60 frames per second on the games lab computers. If the frame rate is less than 60, the game will not run smoothly, which would be detrimental to the game-play and feel of the game. Overall input delay from controller to screen should remain below 200ms so that it does not result in a distraction to the player \cite{Leadbetter09}.

\subsection{Resources and constraints}
In order to setup and test various possible control schemes, two physical game controllers (such as those used by the Xbox 360) are necessary. The Unity game engine (free) is required to develop the game. Other than that there are no special resource development requirements apart from the standard computer hardware needed to develop any software. From the user's perspective, the game may require a game controller in order to appropriately support two local players. In addition to this, the split-screen nature of the local two-player gameplay means that screen real estate might be an issue on smaller displays, potentially warranting the recommendation that players use larger screens, or two different screens attached to a single computer.
\smallskip\\
With regards to the hardware required to allow the game to run smoothly, the game will not be particularly complicated and will easily run on any fairly modern CPU while keeping memory requirements comparatively small. The graphical style of the game may require decent graphics processing hardware but since this can be modified at run time, multiple levels of graphical fidelity will be supported to allow the game to run on older hardware.

\subsection{Feasibility}
The scope is realistic within the given time and given the co-operation of all team members. However, since some aspects of the game (animations, art and sound) will be outsourced, the feasibility of the end product being as polished as desired is a concern. Additionally, while the SMART goals are well defined and reachable, whether or not the game is fun and whether it feels good to play is less well defined and as such may not be entirely feasible, certainly it is not feasible that such a game will be totally enjoyed by all players.
\smallskip\\
 Overall, if the risks stated above are correctly accounted for and handled, it is very reasonable that this project will result in a quality game that the average user will find enjoyable.
\newpage

\begin{thebibliography}{9} 
	\bibitem{Leadbetter09} Leadbetter, R. (2009). Console Gaming: The Lag Factor. [online] Eurogamer.net. Available at: \\http://www.eurogamer.net/articles/digitalfoundry-lag-factor-article?page=2 [Accessed 29 Jul. 2015]. 
\end{thebibliography}
\end{document}
